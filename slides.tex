\documentclass[aspectratio=169,t]{beamer}
\usepackage[utf8]{inputenc}
\usepackage[T1]{fontenc}
\usepackage[export]{adjustbox}
\usepackage{amssymb}


\title{CAMI2: Pathogen Detection}
\date{February, March 12th}
\author[PWD]{Prof. Dr.-Ing. Piotr Wojciech Dabrowski}
\titlegraphic{Bilder/logo.png}

\usepackage{HTWBeamerTemplate/beamerthemeHTW}
%\setbeameroption{show notes on second screen}

\subtitle{The case in a few slides}
\addbibresource{sources.bib}
\begin{document}

\setbeamertemplate{footline}[first]

\begin{frame}[noframenumbering]
    \titlepage
    \begin{textblock}{10}(4.75,15)
			\textcolor{white}{\cite{logo}}
    \end{textblock}
\end{frame}

\setbeamertemplate{footline}[presentationbody] 

\begin{frame}{Case description}
	\begin{itemize}
		\item 32 year old woman
		\item Vomiting, abdominal pain, strong nosebleeds
		\item ILI symptoms and diagnosis 4 and 5 days prior
		\item Returned from Turkey 9 days prior - no wildlife contact or raw meat consumption
	\end{itemize}
	\only<2->{Task: Send taxonomy IDs of pathogens found and taxonomy ID of causative agent based on NGS data from blood sample\\}
	\only<3->{Solution: A highly pathogenic virus!\\}
	\only<4->{...so why a virus for the first pathogen detection challenge?\\}
	\only<5->{...and what is special about real samples?}
\end{frame}

\begin{frame}{Is pathogen detection special?}
	\begin{tikzpicture}
		\node[inner sep=0pt] (hay) at (0,0)
			{\only<1-2>{\includegraphics[width=6.5cm]{Bilder/haystack.jpg}}\only<3->{\includegraphics[width=6.5cm]{Bilder/patient.jpg}}};
		\only<2->{
			\node[inner sep=0pt] (needles) at (7.5,0)
				{\only<2-3>{\includegraphics[width=6.5cm]{Bilder/needlestack.jpg}}\only<4->{\includegraphics[width=6.5cm]{Bilder/needleshoe.jpg}}};
			\draw[->,thick] (hay.east) -- (needles.west);
		}
		\only<1-2>{\draw node[xshift=-10, yshift=10] at (hay.south east) {\cite{hay}};}
		\only<3->{\draw node[xshift=-10, yshift=10] at (hay.south east) {\cite{patient}};}
		\only<2-3>{\draw node[xshift=-10, yshift=10] at (needles.south east) {\cite{needlestack}};}
		\only<4->{\draw node[xshift=-10, yshift=10] at (needles.south east) {\cite{accident}};}
	\end{tikzpicture}
% Challenges:
% * Lots of bacteria, which one is causative? (e.g. sputum with tb and mouth flora, blood sample with virus and skin flora)
% * Sometimes looking for a single read!
% * Unclear symptoms: When nothing obvious, is it the flora bacterium that is causing problems because it's where it's not supposed to be, or it is e.g. autoimmune?
% Solutions:
% * Blacklists/common environmental bacteria/common flora (but: can still cause problems in the wrong place, lots of funky case reports)
% * Inclusion of other data sources, e.g. gideon for ranking of hits based on anamnesis
\end{frame}

\begin{frame}{Are viruses easier?}
	\begin{tikzpicture}
		\node[inner sep=0] (vi) at (0,0)
			{\includegraphics[width=0.53\textwidth]{Bilder/virusintegration.png}}; % 8% of human genome! https://commons.wikimedia.org/wiki/File:Integration_of_viral_DNA_into_host_genome.png
		\node[inner sep=0] (patho) at (8,0)
			{\only<2->{\includegraphics[width=0.26\textwidth]{Bilder/patholive.jpg}}}; % https://www.biorxiv.org/content/10.1101/402370v1 
		\draw node[xshift=-10, yshift=0] at (vi.south east) {\cite{virus}};
		\only<2->{\draw node[xshift=-10, yshift=0] at (patho.south east) {\cite{patholive}};}
	\end{tikzpicture}
% Extra challenges:
% * Very small genome size!
% * 8% of human genome are defunct retroviruses
\end{frame}

\begin{frame}{Privacy concerns}
	\begin{tikzpicture}
		\node[inner sep=0] (camels) at (0,0)
			{\only<2->{\includegraphics[height=0.7\textheight]{Bilder/camels.jpg}}}; 
		\node[inner sep=0] (face) at (5,0)
			{\only<3->{\includegraphics[width=0.3\textwidth]{Bilder/face.jpg}}};
		\node[inner sep=0] (collage) at (10,0)
			{\only<4->{\includegraphics[width=0.3\textwidth]{Bilder/collage.jpg}}};
		\only<4->{\draw[->,thick] (face.east) -- (collage.west);}
		\only<2->{\draw node[xshift=-10, yshift=10] at (camels.south east) {\cite{camels}}};
		\only<3->{\draw node[xshift=-10, yshift=10] at (face.south east) {\cite{face}};}
	\end{tikzpicture}
\end{frame}

\begin{frame}[allowframebreaks]{Sources}
    \printbibliography
\end{frame}

\include{lizenz}

\end{document}
